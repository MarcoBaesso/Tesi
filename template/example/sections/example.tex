
\subsection{Testo}
\LaTeX \\
\textbf{testo in grassetto} \\
\textit{testo in corsivo} \\
\underline{testo sottolineato} \\
\texttt{testo in monospace} \\
\uppercase{testo in maiuscolo} \\
Questo testo non dovrebbe stare qui..\footnote{ma a piè pagina} \\



\subsection{Elenco puntato}
\textbf{Titolo elenco puntato:}
\begin{itemize}
\item Primo Punto;
\item Secondo Punto;
\item Terzo Punto;
\item
	\begin{itemize}
	\item Primo Punto;
	\item Secondo Punto;
	\item Terzo Punto.
	\end{itemize}
\end{itemize}

\subsection{Tabella}
\begin{longtable}{|>{\centering}p{6cm}| >{\centering}m{3cm}| >{\centering}m{3cm}|}
    \hline
    \multicolumn{1}{|c|}{\textbf{Header 0}} &
    \multicolumn{1}{c|}{\textbf{Header 1}} &
    \multicolumn{1}{c|}{\textbf{Header 2}}\\ %\tabularnewline 
      \hline
        Cella 0.0 & Cella 1.0 & Cella 2.0 \tabularnewline
		Cella 0.1 & Cella 1.1 & Cella 2.1 \tabularnewline
		Cella 0.2 & Cella 1.2 & Cella 2.2 \tabularnewline
      \hline
    \caption{Tabella d'esempio}
    \label{tab: Tabella d'esempio}
  \end{longtable}

\subsection{Immagine}
\begin{figure}[H]
\centering % per centrare l'immagine (opzionale)
\includegraphics[width=200px]{img/example.png}
\caption{Immagine d'esempio}
\label{fig:immagine esempio}
\end{figure}

\subsection{UseCase}
Utente autenticato interagisce con il sistema dopo essersi identificato al sistema.
\begin{figure}[H]
  \centering  
  \includegraphics[scale=0.95]{img/ucsample.pdf}
  \caption{Caso d'uso 2 (UC2) - Utente autenticato}\label{fig:uc_utente_loggato} 
\end{figure}
\begin{itemize}
	\item \textbf{Attori}: utente autenticato;
	\item \textbf{Scopo e descrizione}: il sistema gestisce le richieste di un utente che si è identificato; 
	\item \textbf{Pre-condizioni}: l'utente ha effettuato il login e il sistema è in attesa di input da parte dell'utente;
	\item \textbf{Post-condizioni}: il sistema ha gestito le richieste dell'utente(voti, suggerimenti, commenti) ed è pronto per nuove operazioni;
	\item \textbf{Flusso principale degli eventi}:
		\begin{enumerate}
		\item l'utente può dare uno o più voti ad un progetto (UC2.1);
		\item l'utente può suggerire un progetto ad utenti che potrebbero essere interessati (UC2.2);
		\item l'utente può lasciare un commento al progetto al fine di dare un consiglio al proponente (UC2.3);
		\item l'utente può gestire le informazioni del proprio profilo (UC2.4).
		\end{enumerate}
\end{itemize}