In questa sezione vengono riportati i diagrammi di Gantt che forniscono una visione a preventivo del progetto a me commissionato. Nel primo diagramma si ha la visione della pianificazione delle attivit\`{a} prima di intraprendere il progetto.

Questo \`{e} il diagramma che \`{e} stato modificato in relazione alla scelte implementative dell\textquoteright{}azienda.

Un diagramma di Gantt a preventivo, come quelli presentati, si dimostra utile per capire quali sono le dipendenze delle attivit\`{a}, la loro durata (meglio con un diagramma di Pert), le scadenze. I diagrammi di Gantt a consuntivo possono essere utilizzati come strumento di misurazione dell\textquoteright{}abilit\`{a} di un project Manager nella pianificazione dei progetti, tale strumenti quindi possono essere utilizzati al fine di mirare al miglioramento continuo in logica PDCA.

\paragraph{Ciclo di vita}
Il modello di ciclo di vita dell\textquoteright{}azienda rientra tra le metodologie agili. L\textquoteright{}azienda settimanalmente produce il cosidetto \textit{``sprint planning\textquoteright{}\textquoteright{}} ovvero pianifica le attivit\`{a} da svolgere nella settimana successiva. A questa fase si susseguono le attivit\`{a} di verifica pianificate, sia relative allo stato di avanzamento del prodotto sia relative alla qualit\`{a} del prodotto. In relazione al ciclo di vita adottato dall\textquoteright{}azienda ovviamente con gli stagisti ha adottato lo stesso approcio fornendo sul mio progetto revisioni eseguite con una frequenza pari a 2 settimane. Il progetto da me svolto per le caratteristiche presentate si \`{e} concordato con il modello dell\textquoteright{}azienda per quanto riguarda i processi di revisione e di comunicazione; tuttavia \`{e} stato indispensabile adottare per il progetto un ciclo di vita incrementale, nello specifico sono stati pianificati e poi mantenuti 3 incrementi. La natura del ciclo adottata \`{e} stata incrementale proprio a causa del modello adottato dall\textquoteright{}azienda. Infatti il dettaglio dei requisiti del progetto, sono emersi non dall\textquoteright{}inizio ma durante lo stage e quindi durante il ciclo di vita stesso. Il progetto quindi ha prodotto un software la cui vita \`{e} stata animata da baseline successive.