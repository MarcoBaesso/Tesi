\subsection{Strumenti e tecniche di verifica}
Gli strumenti utilizzati sono forniti da Eclipse; in particolare � stato pesantemente utilizzato il debugger, il quale facilita la visione dei contenuti delle varie strutture dati a tempo di esecuzione ed evidenzia per il \textit{\textbf{test-case}} in questione i dati prodotti permettendo in caso di errore di indagare su quali componenti occorre applicare delle correzioni immediate o comunque da sottoporre a correzione in un momento successivo e in quest'ultimo caso segnalando un ticket attraverso uno strumento messo a disposizione dall'azienda \textit{RedMine}. Per la verifica sono state svolte le seguenti 2 tecniche:
\begin{itemize}
\item \textbf{analisi statica}: realizzata dagli strumenti di sviluppo utilizzati che evidenziano eventuali errori commessi, che il flusso di controllo possa essere raggiunto e che i dati non assumano valori illegali o senza senso;
\item \textbf{analisi dinamica}: sono stati pianificati i test di unit� per ogni metodo, indicando nel documento di qualifica del prodotto, identificativo del test e relativi esiti. Sono stati definiti i casi di prova per ogni metodo analizzando il codice dei metodi quindi sono stati effettuati test di tipo white-box, seguendo per ognuno il debugging. Per i metodi aggiunti dalla progettazione e dal debugging sono stati ricavati i flussi di dati e di chiamata al fine di integrare le componenti con la pi� scarsa possibilit� di \textit{\textbf{errors}}; lo strumento utilizzato per lo svolgimento dei test di unit� e integrazinoe � stato JUnit. Le procedure di verifica svolte includevano la seguente \textit{\textbf{checklist}}:
	\begin{enumerate}
	\item \textbf{verifica strutture dati}: sono state controllate le strutture dati utilizzate come tipo di ritorno di un metodo; sono stati controllati gli oggetti immagazzinati in \textit{\textbf{iteratori}}, prevalentemente come risultato di query;
	\item \textbf{inserimento dei dati}: � stato verificata la corretta interazione del software con l'utente e il comportamento in caso di errori in inserimenti dati e il comportamento in caso di inserimenti corretti; � stato controllato anche l'inserimento dei dati nel database e il comportamento in caso di \textit{\textbf{commit}} o \textit{\textbf{rollback}} di transazioni;
	\end{enumerate}
\end{itemize}

\subsection{Collaudo finale}
Per il collaudo finale sono stati eseguiti dapprima i test di sistema pianificati, ovvero un test di sistema per ogni requisito; si tratta di test che verificano dinamicamente se i requisiti fissati sono stati soddisfatti o meno; l'esito complessivo di tali test � stato di esito positivo. Il passo successivo � stata l'accettazione da parte dell'azienda. Nell'ultimo giorno di stage � stato organizzato un'incontro che ha visto coinvolti me, il mio tutor e il titolare dell'azienda. L'oggetto di questo incontro consisteva nella dimostrazione delle modifiche applicate a OnePointProject e una prova di tutte le funzionalit� da me aggiunte. L'esito � stato positivo e il prodotto � stato quindi accettato.