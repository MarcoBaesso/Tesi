In questa sezione vengono elencate le regole e procedure stabilite per codificare ci� che � stato previsto dalla progettazione.
\subsection{Procedure}
Per la codifica � stata seguita la seguente procedura:
\newcommand{\circleditem}[1]{\tikz[baseline=-0.4em] \draw node[draw,circle,inner sep=0,align=center] {\scriptsize \makebox[3mm]{#1}};}
\begin{enumerate}[label=\protect\circleditem{\theenumi}]
\item ciclare sugli incrementi progettati;
\item debugging per identificare le componenti che si integrano con l'incremento pianificato e progettato;
\item codifica;
\item test di unit�;
\item applicazione di commenti al codice;
\item riportare nel changelog le modifiche applicate.
\end{enumerate}
A incremento codificato � necessario procedere con i test di integrazione.\\

Procedura di modifica architettura:
\begin{enumerate}[label=\protect\circleditem{\theenumi}]
\item modifica classi Model;
\item modifica file con estensione *.opt.xml al fine di aggiornare le relazioni del database;
\item modifica dei file con estensione *.oxf.xml al fine di modificare le View necessarie;
\item modifica dei file di script che gestiscono gli eventi;
\item modifica delle classi Java, che svolgono la funzionalit� di handler degli eventi;
\item modifica dell'application logic.
\end{enumerate}


\subsection{Regole}
Le regole principali che sono state seguite sono le seguenti:
\begin{enumerate}[label=\protect\circleditem{\theenumi}]
\item in accordo con la progettazione non violare le dipendenze previste dall'architettura;
\item adottare il principio zero warnings;
\item limitare la complessit� ciclomatica ad un livello pari a 5;
\item il codice deve presentare l'indentazione adottata dall'applicativo;
\item 1 riga = 1 istruzione;
\item non violare regole di codifica adottate dall'applicativo.
\end{enumerate}