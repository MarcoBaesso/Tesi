In questa sezione vengono elencati gli stumenti di sviluppo, le regole e le procedure stabilite per codificare ci� che � stato previsto dalla progettazione.

\subsection{Strumenti di sviluppo}
L'azienda ha imposto e predisposto l'installazione di Eclipse Juno; questo pu� essere utilizzato per la produzione di software di vario genere fornendo un completo \textit{\textbf{IDE}} per il linguaggio Java. La piattaforma di sviluppo � incentrata sull'uso di \textit{\textbf{plug-in}} che possono essere da chiunque sviluppati. Alla versione installata � stato aggiunto il plug-in Eclipse Subversive, al fine di versionare il codice. Il repository � stato creato sui server aziendali. Sullo stesso sono stati aggiunti anche i documenti tecnici sviluppati durante la fase di manutenzione.

\subsection{Procedure}
Per la codifica � stata seguita la seguente procedura:
\newcommand{\circleditem}[1]{\tikz[baseline=-0.4em] \draw node[draw,circle,inner sep=0,align=center] {\scriptsize \makebox[3mm]{#1}};}
\begin{enumerate}[label=\protect\circleditem{\theenumi}]
\item ciclare sugli incrementi progettati;
\item debugging per identificare le componenti che si integrano con l'incremento pianificato e progettato;
\item codifica;
\item test di unit�;
\item applicazione di commenti al codice;
\item riportare nel \textit{\textbf{changelog}} le modifiche applicate.
\end{enumerate}
A incremento codificato � necessario procedere con i test di integrazione. Non essendo stata comunicata una prassi aziendale riguardante la documentazione del codice si sono scelti i seguenti metodi di documentazione:
\begin{itemize}
\item \textbf{commenti interni al codice}: le righe di codice che potrebbero risultare di comprensione non intuitiva o di una certa complessit� devono essere commentate, fornendo dettagli tecnici;
\item \textbf{architettura di dettaglio}: oltre all'identificazione delle classi da modificare e ai diagrammi delle classi relativi alle integrazioni occorre fornire una documentazione sui metodi da implementare munendoli di firma e di una descrizione sulle funzionalit� che essi compiono.
\end{itemize}


Procedura di modifica architettura:
\begin{enumerate}[label=\protect\circleditem{\theenumi}]
\item modifica classi Model;
\item modifica file con estensione *.opt.xml al fine di aggiornare le relazioni del database;
\item modifica dei file con estensione *.oxf.xml al fine di modificare le View necessarie;
\item modifica dei file di script che gestiscono gli eventi;
\item modifica delle classi Java, che svolgono la funzionalit� di handler degli eventi;
\item modifica dell'application logic.
\end{enumerate}


\subsection{Regole}
Le regole principali che sono state seguite sono le seguenti:
\begin{enumerate}[label=\protect\circleditem{\theenumi}]
\item in accordo con la progettazione non violare le dipendenze previste dall'architettura;
\item adottare il principio \textit{\textbf{zero warnings}};
\item limitare la \textit{\textbf{complessit� ciclomatica}} ad un livello pari a 5;
\item il codice deve presentare l'indentazione adottata dall'applicativo;
\item 1 riga = 1 istruzione;
\item non violare regole di codifica adottate dall'applicativo.
\end{enumerate}