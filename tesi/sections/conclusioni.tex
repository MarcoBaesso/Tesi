Lo stage determina il passo intermedio tra la scuola e il mondo del lavoro e solo con questa esperienza o comunque altre attivit� lavorative � possibile integrarsi in questo ambiente. In questa esperienza, la prima presso un'azienda informatica, sono entrato in diretto contatto con le difficolt� che animano un'azienda quotidianamente. I problemi non sono strettamente di natura informatica, ma anche di gestione del personale e organizzazione del lavoro. La struttura dell'azienda non � predisposta per sviluppo di progetti, in modo tale quindi da dedicare le risorse alla sole attivit� di progetto; l'azienda � caratterizzata da risorse che alternano le mansioni di una risorsa dall'analisi allo sviluppo del software, dall'incontro con i clienti alla manutenzione di software ad essi venduti. Da questo ne ricavo che � indispensabile avere un'organizzazione delle risorse e dell'ambiente di lavoro al fine di fissare le funzionalit� competenti una deternminata risorsa, le responsabilit�, le procedure e le regole. Dallo stage ho appreso l'importanza delle conoscenze acquisite grazie all'universit�, ma ho soprattutto capito che la laurea � un traguardo importante, ma in questo settore ci� che vince � l'esperienza o meglio le best practice. Per quanto riguarda il progetto di stage sono rimasto soddisfatto del risultato ottenuto, inoltre la manutenzione di software ha come pregio la possibilit� di analizzare le scelte architetturali di altri team, permettendo quindi di valutare le scelte implementative, pensando a strategie diverse di sviluppo; il tutto per un laureando � molto utile perch� permette di imparare da altri puntando necessariamente al miglioramento continuo. Quindi la mia esperienza di stage si � alternata da fasi monotone di debugging, anche un po noiose, ma indispensabili per apprendere l'architettura del sofware a fasi di modifica dell'applicativo che hanno permesso di portare avanti il ciclo di vita del software di OPP.