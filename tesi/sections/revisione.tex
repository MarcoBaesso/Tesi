Per il progetto sono state fissate delle milestone al termine delle quali � stata concordata un'operazione di controllo. L'introduzione delle revisioni nel piano di progetto � stata fatta in accordo agli standard aziendali, anche se dall'azienda sono eseguite settimanalmente piuttosto che alla fine di milestone precise. Ogni revisione � stata svolta nella sala riunioni dell'azienda in cui sono presenti un proiettore, una lavagna, un tavolo, e dei computer; lo strumento che � stato deciso di utilizzare � il proiettore, quindi la milestone doveva concludere con la realizzazione di una presentazione in cui figuravano i contenuti essenziali.\\
La prima revisione � stata eseguita al termine della fase di lavoro in comune; in questa revisione sono stati divisi i contenuti tra me e il collega stagista Petrin; in questa revisione sono state spiegate le scelte implementative di OPP, note all'azienda solo parzialmente e questa almeno da parte mia � stata una sorpresa nel senso che il programma nel dettaglio non era conosciuto, ma � stato scelto come la miglior soluzione che l'azienda potesse adottare e quindi adattare ai requisiti. OPP in realt� essendo privo di documentazione, da quello che ho potuto sperimentare richiede tempo nelle modifiche da applicare, proprio perch� non conoscendo dall'inizio le componenti da modificare occore identificarle, ma non sempre si riesce a identificarle tutte fin dall'inizio e quindi nello sviluppo si arriva ad uno stato incerto che potrebbe rivedere delle modifiche alla progettazione. Questo fatto � stato presentato all'azienda, ma non gli � stata data molta importanza. \\
Alla seconda revisione sono stati presentati per la parte relativa al mio progetto i casi d'uso al fine di chiarire se avevo compreso quello che il software doveva realizzare; questa revisione � stata molto utile perch� a me sono stati chiariti dei punti un p� oscuri, mentre l'azienda ha riflettuto meglio sul da farsi e ha mosso dei ticket in cui venivano applicate delle modifiche e aggiunte ai requisiti previsti. Lo strumento dei casi d'uso con \textit{\textbf{UML}} non � applicato dall'azienda, ma lo ha trovato utile e intuitivo. \\
La terza revisione � stata complessa da presentare, per il fatto che le scelte progettuali richiedevano per la maggior parte la modifica di metodi gi� sviluppati. La presentazione � stata divisa in 3 parti:
\begin{description}
\item nella prima parte venivano indicate le componenti che l'applicativo non presentava e che dovevano essere aggiunte;
\item nella seconda parte sono state presentate le classi e i rispettivi metodi che richiedavano modifiche nel contenuto o nella firma; una modifica della firma presentava una catena di altre modifiche per tutte le classi che si interfacciavano con il metodo da modificare, per questa ragione le modifiche di questo tipo di questo tipo si � cercato di ridurle al massimo;
\item alla fine � stato presentato il tracciamento componenti-requisiti in cui veniva specificata la ragione che ha portato a quel tipo di progettazione.
\end{description}

L'ultima revisione coincideva con l'accettazione del software. A questa revisione � stata presentata la qualifica del prodotto, o meglio gli esiti dei test eseguiti, e l'esecuzione dei test di sistema. L'esito complessivo � stato per l'azienda soddisfacente.