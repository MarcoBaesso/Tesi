Nel documento la prima occorrenza delle parole presenti nel glossario vengono evidenziate con la formattazione in grassetto e corsivo.\\ \\

\Huge A
\normalsize
\begin{itemize}
\item \textbf{ad hoc}: ovvero appropriato, specificatamente realizzato.
\item \textbf{attivit�}: organizzazione di compiti.
\item \textbf{attributi}: propriet� di una certa classe o tabella.
\end{itemize}

\Huge B
\normalsize
\begin{itemize}
\item \textbf{baseline}: pianificazione del software ad una certa versione.
\end{itemize}

\Huge C
\normalsize
\begin{itemize}
\item \textbf{componenti}: elemento definito con funzionalit�, responsabilit� e interfacce specifiche.
\item \textbf{CQT}: si tratta di un software prodotto dall'azienda RDS-NORDEST che � in fase di manutenzione.
\item \textbf{CRUD}: create,read,update and delete.
\item \textbf{check-in}: in OPP operazione che rende visibile a tutti gli utenti le modifiche effettuate su un determinato progetto.
\item \textbf{collection\_activity}: tipo definito in OPP per indicare un'attivit� che racchiude un sottoinsieme di sottoattivit� di tipo standard o collection.
\item \textbf{changelog}: registro delle modifiche.
\item \textbf{complessit� ciclomatica}: nota anche come complessit� di McCabe; si tratta di una metrica software utilizzata per misurare la complessit� di un programma; misura direttamente il numero di cammini linearmente indipendenti attraverso il grafo di controllo di flusso.
\item \textbf{commit}: esito positivo nell'esecuzione di una transazione che porta il database ad un nuovo stato.
\item \textbf{checklist}: lista di controllo ovvero un qualsiasi elenco esaustivo di cose da fare o da verificare per eseguire una determinata attivit�.
\end{itemize}

\Huge D
\normalsize
\begin{itemize}
\item \textbf{debugging}: attivit� che consiste nell'esecuzione del software guidata dal verificatore.
\item \textbf{dominio}: termine utilizzato nella fase di analisi dei requisiti per indicare l'ambiente con il quale il sistema deve integrarsi.
\end{itemize}

\Huge E
\normalsize
\begin{itemize}
\item \textbf{ennuple}: l'insieme dei valori in una tabella di un database.
\end{itemize}

\Huge F
\normalsize
\begin{itemize}
\item \textbf{form}: interfaccia che consente all'utente di inserire dati.
\item \textbf{firma}: prototipo del metodo.
\item \textbf{factory}: ovvero creato per la creazione di altri oggetti.
\end{itemize}

\Huge G
\normalsize
\begin{itemize}
\item \textbf{GDO}: ovvero grande distribuzione organizzata.
\item \textbf{Gantt}: il diagramma di Gantt � lo strumento pi� utilizzato tanto nella fase operativa, quanto in quella di controllo; si tratta di un diagramma cartesiano, ovvero, una rappresentazione grafica bidimensionale: sulle ascisse (lungo l'asse orizzontale) viene riportata la variabile temporale, mentre lungo le ordinate (asse verticale) soni indicate le attivit� nella quali � stato scomposto il progetto.
\end{itemize}

\Huge H
\normalsize
\begin{itemize}
\item \textbf{Ho.Re.Ca.}: azienda che si occupa della distribuzione di alimentari.
\item \textbf{Hibernate}: � una piattaforma middleware open source per lo sviluppo di applicazioni Java, attraverso l'appoggio al relativo framework, che fornisce un servizio di object-relational mapping (ORM) ovvero gestisce la persistenza dei dati sul database attraverso la rappresentazione e il mantenimento su database relazionale di un sistema di oggetti Java.
\end{itemize}

\Huge I
\normalsize
\begin{itemize}
\item \textbf{impegno}: indica lo sforzo assegnato ad una risorsa per una certa attivit�.
\item \textbf{IDE}: componente integrato a editor di testo utilizzata per fornire aiuti di sintassi al programmatore.
\item \textbf{iteratori}: un iteratore � un oggetto che consente di visitare tutti gli elementi contenuti in un altro oggetto, tipicamente un contenitore.
\end{itemize}

\Huge J
\normalsize
\begin{itemize}
\item \textbf{Java}: linguaggio di programmazione orientato agli oggetti.
\end{itemize}

\Huge K
\normalsize
\begin{itemize}
\item \textbf{know-how}: identifica le conoscenze e le abilit� operative necessarie per svolgere una determinata attivit� lavorativa.
\end{itemize}

\Huge L
\normalsize
\begin{itemize}
\item \textbf{LIMS}: software usato nei laboratori d'analisi per la gestione integrata di molteplici tipi di dati e processi.
\item \textbf{lock}: meccanismo di sincronizzazione per limitare l'accesso ad una risorsa condivisa.
\end{itemize}

\Huge M
\normalsize
\begin{itemize}
\item \textbf{milestone}: traguardi intermedi previsto nello svolgimento di un progetto; rappresenta un'attivit� con nome e durata pari a 0.
\item \textbf{macrocomponenti}: componente non terminale che � formato da sottocomponenti; termine utilizzato per dare una definizione di prodotto ad alto livello. 
\item \textbf{metodo}: operazione definita in una classe che pu� essere eseguita sugli oggetti istanze di tale classe.
\item \textbf{MsSQL}: DBMS relazionale prodotto da Microsoft.
\end{itemize}

%\Huge N
%\normalsize
%\begin{itemize}
%\item \textbf{}
%\end{itemize}

\Huge O
\normalsize
\begin{itemize}
\item \textbf{oggetto}: con oggetto nell'ambito della programmazione, si intende nella maniera pi� generica una regione di memoria allocata; poich� i linguaggi di programmazione usano variabili per accedere agli oggetti, i termini oggetto e variabile sono spesso usati in alternativa tra loro, in ogni caso finch� un'area di memoria non � allocata nessun oggetto pu� esistere.
\end{itemize}

\Huge P
\normalsize
\begin{itemize}
\item \textbf{Pert}: il diagramma reticolare di PERT descrive la sequenza cronologica delle azioni pianificate per il completamento di un progetto nel suo complesso; esso rappresenta graficamente il piano d'azione.
\item \textbf{project manager}: ruolo di gestione operativa; tale figura � il responsabile unico della valutazione, pianificazione, realizzazione e controllo di un progetto.
\item \textbf{PDCA}: modello studiato per il miglioramento continuo della qualit�; noto anche come ciclo di Deming.
\item \textbf{proxy}: si tratta di componenti che si interpongono tra altri al fine di monitorare o eventualmante limitare le operazioni che li coinvolgono.
\item \textbf{plug-in}: programma non autonomo che interagisce con un altro programma per ampliarne le funzioni.
\end{itemize}

\Huge Q
\normalsize
\begin{itemize}
\item \textbf{query}: l'interrogazione su un database per compiere determinate operazioni sui dati: inserimento, modifica, lettura.
\end{itemize}

\Huge R
\normalsize
\begin{itemize}
\item \textbf{risorsa}: elemento utile per l'esecuzione di un'attivit�.
\item \textbf{run-time}: indica il momento in cui un programma viene eseguito.
\item \textbf{repository}: si tratta di un database centralizzato nel quale risiedono individualmente tutti i componenti di ogni baseline nella loro storia completa.
\item \textbf{rollback}: rappresenta il caso del fallimento di una transazione che riporta il database allo stato precedente l'inizio della transazione.
\end{itemize}

\Huge S
\normalsize
\begin{itemize}
\item \textbf{SourceForge}: SourceForge � una piattaforma e un sito web che fornisce gli strumenti per portare avanti un progetto di sviluppo software in modo collaborativo tra gli sviluppatori.
\item \textbf{standard\_activity}: tipo definito in OPP per indicare un'attivit� appunto di tipo standard, caratterizzata da nome, data di inizio e di fine, duratave risorse.
\end{itemize}

\Huge T
\normalsize
\begin{itemize}
\item \textbf{tag}: si tratta di parole chiave che associano un'informazione che descrive l'oggetto; i tag rendono possibile la ricerca di informazioni.
\item \textbf{transazioni}: sequenza di operazioni che pu� concludersi con un successo o un insucceso; una transazione gode delle cosiddette propriet� ACID:
\begin{description}
\item {atomicit�}: la transazione � indivisibile nella sua esecuzione e la sua esecuzione deve essere totale o nulla, non sono ammesse esecuzioni parziali;
\item {coerenza}: quando inizia una transazione il database si trova in uno stato coerente; quando la transazione termina il database deve essere in un altro stato coerente, ovvero non devono verificarsi contraddizioni tra i dati archiviati nel DB;
\item {isolamento}: ogni transazione deve essere eseguita in modo isolato e indipendente dalle altre transazioni;
\item {durabilit�}: ovvero persistenza; i cambiamenti apportati non dovranno essere pi� persi.
\end{description}
\item \textbf{test-case}: insieme di condizioni sotto le quali un determinato test determina se un metodo risponde correttamente o meno.
\end{itemize}

\Huge U
\normalsize
\begin{itemize}
\item \textbf{UML}: si tratta di un linguaggio di modellazione e specifica basato sul paradigma object-oriented; � un linguaggio utilizzato per descrivere soluzioni analitiche e progettuali in modo sintetico e compresibile a un vasto pubblico.
\end{itemize}

\Huge V
\normalsize
\begin{itemize}
\item \textbf{value proposition}: si tratta di un elemento fondamentale del business model di un'azienda. Per value proposition si intende l'insieme di benefici per cui un utente gode acquisendo un determinato prodotto dell'azienda.
\item \textbf{VIDA}: si tratta di un software in via di sviluppo dall'azienda RDS-NORDEST.
\end{itemize}

\Huge W
\normalsize
\begin{itemize}
\item \textbf{WBS}: ovvero WorkBreakdownStructure, � l'elenco di tutte le attivit� di un progetto.
\end{itemize}

\Huge X
\normalsize
\begin{itemize}
\item \textbf{XML}: si tratta di un linguaggio di markup basato su un meccanismo sintattico che consente di definire e controllare il significato degli elementi contenuti in un documento.
\end{itemize}

%\Huge Y
%\normalsize
%\begin{itemize}
%\item \textbf{}
%\end{itemize}

\Huge Z
\normalsize
\begin{itemize}
\item \textbf{zero warnings}: si tratta di una regola adottata da Gerard J. Holzmann, in cui spiega che il codice va compilato da subito usando il compilatore nel modo pi� restrittivo.
\end{itemize}




